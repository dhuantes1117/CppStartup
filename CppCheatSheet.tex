\documentclass[12pt]{article}
\usepackage{dsfont}
\usepackage{minted}
\usepackage{cite}
\usepackage{amsmath}
\usepackage{amsfonts}
\usepackage{mathbbol}
\usepackage{graphicx}
\usepackage{float}
\usepackage{siunitx}
\usepackage{hyperref}
\usepackage{listings}
\usepackage[cm]{fullpage}
\usepackage{layout}
\usepackage[dvipsnames]{xcolor}
\lstset{language=C++,
  basicstyle=\ttfamily,
  keywordstyle=\color{blue}\ttfamily,
  stringstyle=\color{red}\ttfamily,
  commentstyle=\color{CadetBlue}\ttfamily,
  morecomment=[l][\color{magenta}]{\#}
}

\begin{document}
\title{C++ cheat sheet}
\author{Daniel Huantes}
\maketitle
\section{Introduction}

\section{Basics}
  \subsection{Starting}
    \begin{itemize}
      \item All C++ files need to be saved in a file ending in .cpp
      \item All 'header' files, or files containing information
      including class definitions and function definitions should be
      stored in .hpp or .h files.
      \item Header files should include ONE class definition per, with
      the class sharing the name of the file (minus the .hpp)
      \item C++ files need a method named 'main' in order to
      execute. The compiler looks for a method with this name and
      begins executing from its first line, NOT the first line of the
      file like in python
    \end{itemize}
  \subsection{Including}
    \begin{lstlisting}
      //equivalent to importing in python
      \#include </*libraryName*/>
      \#include </*path/to/file.hpp*/>
      \#include "/*local header file*/"

      \#include <set>
      \#include <map>
      \#include <cmath>
      \#include <chrono>
      \#include <vector>
      \#include <iostream>
      \#include <libField/Field.hpp>
      \#include <libInterpolate/Interpolate.hpp>
      \#include "geometry.h"

    \end{lstlisting}
  \subsection{Common Type Declarations}
    \begin{lstlisting}
      // decimal numbers
      double a  = 321.044; 
      float b = 1.21;
      //whole numbers
      int c     = 45;
      long d = 323e7;  
      // true/false
      bool d = true;
      bool d = 3 < 4;
      // strings
      std::string words = "blah blah blah";
      std::string more_words = "foo bar BAZ!"
      // assumed type
      auto something = 23;
    \end{lstlisting}
  \subsection{Common Functions}
    \begin{lstlisting}
      // printing
      std::cout << "Don't forget the newline!" << "\n";
      std::cout << "This is fine as well!\n";
      std::cout << "Now " << "you're " << "being " << "ridiculous\n";
      std::cout << "numbers: {" << 1 << ", " << 0.243 << "}\n";
      // to string
      float b = 1.21;
      std::string interp = std::to_string(b) + " look at that!\n";
      // concatenation
      std::string stringAddition = "Run " + "strings" + " together";
      // mathy stuff
      // these are non-native and are included in cmath
      // pow takes two arguments, x and y and returns x^y
      double IV = pow(25, 2);
      // the log function is natural log, or ln. If you want log base
      // 10 reevaluate your life
      double natural = log(0.5);
      // the square root or sqrt function will always give the
      // positive root
      double irrational = sqrt(2);
    \end{lstlisting}
  \subsection{Common Structure syntax}
    \begin{lstlisting}
      for(/*declaration*/; /*test*/; /*increment/increase*/){
         // do something 10 times 
      }
      for(int i = 0; i < 10; i++){
         // do something N times 
      }
      if(/*conditional*/){
        //do if conditional is true
      }
      a = 1;
      if(a < 2){
        //do this if conditional is true
      } else {
        //do this if conditional ISN'T true
      }
      while(/*conditional*/){
        // repeat this until conditional is false
      }
      int n = 0;
      while(n < 10){
        // do this over and over
        n++;
      }
    \end{lstlisting}
  \subsection{Compiling (running)}
      in python compiling/running are interchangable, not in C++
      C++ 'compiles' source code (human language) and sends it 
      to 'machine code' (binary) that is stored in an
      unreadable, but executable file

      run the following in the command line to compile (replace
      FILENAME with your file's name)
      \begin{minted}{bash}
        \$ g++ FILENAME.cpp -o FILENAME
      \end{minted}
      
      this will generate a file called FILENAME that is executable and
      can be run using
      \begin{minted}{bash}
        \$ ./FILENAME
      \end{minted}

\section{Slightly-less-basics}
  \subsection{Writing to file}
    \begin{lstlisting}
      ofstream output;
      output.open("./path/to/file.txt");
      output << /*Some string or writable thing*/;
      output.close();
    \end{lstlisting}
  \subsection{Plotting}
    \begin{lstlisting}
      ofstream output;
      output.open("./path/to/file.txt");
      //gnuplot style is 'x y' columns
      output << /*Something gnuplot style*/;
      output.close();
      // -e is the commandline tack, and -p is the
      // 'persist' command so it doesn't get destroyed
      system("gnuplot -p -e \'plot \"gnuplotFile.txt\" with lines \'");
      
      //example of y = x^2
      double dx = 0.2;
      std::string buffer = "";
      std::string x, y;
      for(int i = 0; i < 100; i++){
        x = std::to_string(i * dx);
        y = std::to_string(pow(i * dx, 2));
        buffer +=  x + "\t" + y + "\n";
      }
      ofstream output;
      output.open("gnuplotFile.txt");
      //gnuplot style is 'x y' columns
      output << buffer;
      output.close();
      // -e is the commandline tack, and -p is the
      // 'persist' command so it doesn't get destroyed
      system("gnuplot -p -e \'plot \"gnuplotFile.txt\" with lines \'");
    \end{lstlisting}
  \subsection{Function Declaration syntax}
    \begin{lstlisting}
      /*return type*/ /*unique name*/(/*parameters*/){
        // body of function declaration, any math manipulations or
        algorithms go here. Any function with a return type not void
        must have a return statement
      }
      double sum_squared(double x, double y){
        return (x + y) * (x + y);
      }
    \end{lstlisting}
  \subsection{Class Declaration syntax}
    \begin{lstlisting}
      //best practice is to have one class definition per header file
      class /*Name*/{
        public:
          // constructors, public functions, etc
        protected:
          // usable by subclasses
        private:
          // not directly accessible outside of class definition
      };
      // example class for an app to manage Jimmy John's orders
      class Sandwich{
        public:
          // constructors are special 'creation' functions so they
          // share the class name and do not have a return type
          Sandwich(){
          
          }
          Sandwich(int menuNumber){
            if(menuNumber > 17){
              menuID = -1;
            } else {
              menuID = menuNumber;
            }
          }
          string printDescription(){
            return descriptions[menuID - 1];
          }
        protected:
          bool mayo = true;
          bool lettuce = true;
          bool tomato = true;
          bool peppers = false;
        private:
          int menuID;
          string descriptions[] = {/*too lazy to actually type this*/};
      };
    \end{lstlisting}
\section{Even-more-less basic}
  \subsection{Container Declarations}
    \begin{lstlisting}
      /*type*/ /*name*/[/*number*/];
      /*type*/ /*name*/[/*number*/] = {/*same \# of elements*/};
      /*type*/ /*name*/[] = {/*any \# of elements*/};
      double x\_values[n];
      double t\_values[100];
      double ages[3] = {34, 23, 17};
      string names[] = {"Emily", "Dan", "Patrick"};

      std::vector</*typename*/> /*name*/;
      std::vector<int> fibbonacci;
      std::vector<int> threes{3, 3, 3};
      std::vector<n, 1> zeros;
      std::vector<n, 0> ones;

      std::map</*give this type*/, /*get that type*/> /*name*/;
      std::map<int, string> MenuDescriptions = {{1, "THE PEPE: Ham and
      Provolone..."}, 
      {2 "Big John: Roast beef, lettuce, ..."}, 
      {3, "Totally Tuna: It's tuna baby"}};
    \end{lstlisting}
  \subsection{Accessing Container Members}
    \begin{lstlisting}
      /*array name*/[/*index*/]
      /*array name*/[/*index*/] = /*new value*/;

      std::cout << user\_prompts[4];
      x\_values[4] += 3.32;

      /*vector name*/.push\_back(/*some new member, added to end*/);
      // fake constructor for 'Student' class,  objects added to
      // containers don't have to be named
      students.push\_back(Student("Matthew", 8)); 

      /*Map Name*/[/*new key*/] = /*new value*/;
      MenuDescriptions[13] = "JIMMY CUBANO: Bacon, smoked ham...";
   \end{lstlisting} 
  \subsection{Template declarations}
    \begin{lstlisting}
      //add at top of functions to refer to inputs, etc
      template <typename /*stand in name*/>;
      template <typename /*...*/, typename /*...*/>;
      template<typename T, typename S>
      T minimum(T v1, S v2){
        if(v1 - v2 < 0){
          return v2;
        } else {
          return v1
        }
      }

      //add at top of class defs to require a type on declaration
      template<class animalType>
      class Pasture{
        public:
          Pasture(){
          
          }
          Pasture(std::vector<animalType> new_herd){
            for(int i = 0; i < new_herd.size(); i++){
              herd.push_back(new_herd[i]);
            }
          }
          Pasture(animalType animal){
            herd.push_back(animal);
          }
        protected:
          int feedingtime = 500;
        private:
          std::vector<animalType> herd;
      }

    \end{lstlisting}
  \subsection{Reading from a file}
    \begin{lstlisting}
      /*I barely know how to do this, updates welcome*/
      std::ifstream t;
      int length;
      t.open("path/to/file.txt");
      length = t.tellg()
      t.seekg(0, std::ios::end);
      buffer = new char[length];
      t.read(buffer, length)
      t.close();
    \end{lstlisting}
  \subsection{Reading from a file}
  \subsection{Special functions} %lambda, overwriting etc
    \begin{lstlisting}
      //a lambda function is basically an unnamed function
      // auto /*name*/ = [/*vars*/](/*param*/) -> /*ret type*/ {/*return*/};
      auto Tinf = [](double t) -> double {return sqrt(t * 2)};
      // We basically use lambda functions to use the set_f method on a field


      // you can 'override' functions based on the parameters you define
      // different functions with same name using different parameters, and
      // the compiler decides what to use based on arguments
      // often less efficient than templating or "optional" args
      bool even(int n){
        return (n \% 2 == 0);
      }
      bool even(double a){
        return (((int) n) \% 2 == 0);
      }
      bool even(int n, int divisor){
        return (n \% divisor == 0);
      }
    \end{lstlisting}

\section{Clark Galore}
  \subsection{Setting up a Field}
    \begin{lstlisting}
      Field</*some type*/, /*int for dimensions*/>/*Name*/(/*dimensions*/);
      Field<double, 1> T_vs_t(100);

      Field</*some type*/>/*Name*/(/*x dim*/, /*y dim*/);
      Field<double> T_vs_t(10, 20);
    \end{lstlisting}
  \subsection{Setting up a Coordinate System}
    \begin{lstlisting}
      Field</*some type*/, /*# of axes*/>/*Name*/(/*dimensions*/);
      /*Field Name*/.setCoordinateSystem(/*RangeDiscretizer*/);
      // we set the coordinate system using a custom type called a
      // RangeDiscretizer which handles the problem of going from 
      // i -> x and N

      Field<double, 1> T_vs_t(100);
      T_vs_t.setCoordinateSystem(Uniform<double>(0, 5));
    \end{lstlisting}
  \subsection{Range Discretizers}
    \begin{lstlisting}
      Field</*some type*/, /*int for dimensions*/>/*Name*/(/*dimensions*/);

      Field<double, 1> T_vs_t(100);
    \end{lstlisting}
\section{Debugging}
  \subsection{Common fixes}
    \begin{itemize}
      \item Have you tried adding a semicolon?
      \item Did you (re)compile?
      \item Are you compiling a .cpp file with a main method?
      \item Did you declare everything you use?
      \item Are you sure the computer is interpreting your math correctly?
      \item PEMDAS?
      \item Print out intermediate variables to make sure they are correct
      \item Are you including ALL of the libraries you need?
      \item Google it! No one really ever learns to code, they just get
      better at using stack overflow
    \end{itemize}
  \subsection{Less common fixes}
    \begin{itemize}
      \item Are you sure you are respecting namespace? Are you correctly
      \item referring to the variable you intend to? (Did you blow out a
      value?)
      \item Is it passing by reference or value?
      \item Are you including a library one of your included libraries already
    includes?
      \item Are race conditions occuring?
    \end{itemize}
  \subsection{Uncommon fixes}
    \begin{itemize}
      \item Null pointer?
      \item Are you trying to access unallocated memory?
      \item Is there a namespace conflict?
    \end{itemize}
  \subsection{Last-Ditch-Effort fixes}
    \begin{itemize}
      \item Would garbage collection have occurred?
      \item Double imprecision?
      \item How's your prayer life?
    \end{itemize}
\end{document}


